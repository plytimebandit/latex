\documentclass[12pt,oneside]{scrartcl}
\usepackage[utf8]{inputenc}
\usepackage[ngerman]{babel}
\usepackage[babel,german=quotes]{csquotes}
\usepackage[T1]{fontenc}
\usepackage[a4paper, left=4cm, right=2cm, top=3cm, bottom=2.1cm]{geometry}
\usepackage{amssymb}
\usepackage{mathptmx}
\usepackage{amsmath}
\usepackage[scaled=.90]{helvet}
\renewcommand{\familydefault}{\sfdefault}

\usepackage{siunitx}
\AtBeginDocument{\sisetup{
  detect-all, % auto config
  math-sf=\mathrm, % serif in equations
  text-rm=\sffamily, % sans serif in text
  locale = DE, % 1,000.99 -> 1.000,99
  per-mode = symbol, % ms^-1 -> m/s
}}


\begin{document}

\section{Inline}

Show different representations of the number \enquote{123.45}. Sans serif font is used in text and serif inside math environments.

\bigskip

Num: \num{123.45}

Math: $123.45$

\section{Align}

\begin{align}
no formatting &= 123.45 \\
num &= \num{123.45}
\end{align}

\section{Equation}

\begin{equation}
\begin{split}
no formatting = 123.45 \\
num = \num{123.45}
\end{split}
\end{equation}

\section{Multiline}

\begin{multline}
no formatting = 123.45 \\
num = \num{123.45}
\end{multline}

\end{document}
